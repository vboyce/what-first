\documentclass[11pt]{article}
\usepackage[margin=1in, paperwidth=8.5in, paperheight=11in]{geometry}

\usepackage{graphicx}
\setlength\intextsep{0pt}
\pagestyle{empty}
\setlength{\belowcaptionskip}{-10pt}


\usepackage{graphicx}
\bibliographystyle{apalike}
\title{Inferences from distribution of person- and identity- first language}
\author{Veronica Boyce}
\pagestyle{empty}
\begin{document}
	\maketitle

\textbf{Intro}
When referring to people with some property, there's a question of how to express that property. It could be used as a noun (`the Deaf', `the gifted', `manic-depressives'), as an adjective (`gifted children', `autistic children', `depressed people'), or as a post-nominal modifier (`people with BPD', `children with disabilities', `individuals with paraplegia'). These first two methods are referred to as identity-first while the last is referred to as person-first. 

Some people argue that person-first is more respectful because it emphasizes the common humanity of people by having the word person be more prominent (this is clearly based in English-type languages where adjectives are prenominals and other modifier clauses are postnominal) \cite{dunnPersonfirstIdentityfirstLanguage20150202}. In contrast, some disability-rights people prefer identity-first language to describe themselves because they feel that their conditions are inseparable from their identities and not something to be ashamed of \cite{dunnPersonfirstIdentityfirstLanguage20150202}. The APA decided to rule the use of person-first, despite this disagreement. 

There's an empirical question here about the effects of these different constructions. The person-first language promoters make a seductive, but not-well tested claim about how language effects thought.
However, it's (always) too late for whatever pure associations there are because there are distributional pragmatics to reckon with. \cite{gernsbacherEditorialPerspectiveUse2017} counted up the uses of person-first and identity-first language in the academic literature, and finds that person-first language is more common for children versus adults and is more common for less desirable conditions. Thus, more positive traits are more likely to be pre-nominal, resulting in phrases like ``gifted children with autism''. These sorts of distributions may then have paradoxical effects whereby the (supposedly more respectful) person-first language serves to indicate the distastefulness of a condition. 


It seems like two axes are primarily involved here. One is how essentialized the trait is -- is it a core part of the person's identity that doesn't change over time, or an external or incidental effect? Another is valence: is the condition a positive state (ex. giftedness) or a negative one (ex. a disease)? 

\textbf{Proposed model}
Stripped of its social implications, this comes down to set of terms that share some structural similarity with each other. I assume that there is some expectation that similar constructions have similar meanings, and some priors on the traits of different conditions.

We start out with some priors on linguistic constructions (``the X'' might be more essentialized than ``a person with X''), and some priors about the essentialness and valence of some conditions. Then we add observations of language (observed condition/construction) pairs and see how this shifts the beliefs. Notably, this can also lead to beliefs forming about previously unobserved conditions on the basis of the language. 

I may eventually run human-experiments on this, using made-up conditions (`glorpist', `glorpy person', `person with glorpism'), so for this project I want to build a model that I will be able to use as a companion to that. 

I'd like to explore how different priors affect the model and what happens when priors on linguistic construction's meanings conflict with the distributional use. How will this shift the distributional use? How will language shape the beliefs about the badness and inherentness of a new condition? 
\bibliography{person-first.bib}
\end{document}